\chapter{Introduction}

\begin{comment}
The field of autonomous agent technologies has been growing in importance in various aspects of reality, reaching in applications from industrial robots overtaking factories and rescue robots saving lives in areas of natural disasters through autonomous agents on stock markets to a whole range of smart devices. The key common feature connecting all these applications, although to a different extent, is the concept of autonomy. Such autonomous systems "form a general class of self-regulating systems, those that form their own laws of regulation, as
well as regulate their behaviour with respect to these self-made laws." \cite{autonomy} 

%% [NOT FULLY RESOLVED YET]
%% * [Krisy comment 13/01/2019: "plays an important role" is an extremely common thing to read/hear as the first out of a topic]
%% * [Krisy comment 06/02/2019: I added as a todo the recommendation for the fix of this]
%% * [Filip comment 10/02/2019: I also agree, updated]
%% * [Stefan comment 13/01/2019: you start with a statement that "agent technologies plays an ever more important role in various aspects of reality" and the IG people told us that this is bad cause at this point of writing you 1st sentence you have no cred of material analised, Georgon explained to me that to not be shit at 1st sentence like I was on my 1st IG homework you should start with statictics [as a very safe bet and something germans like as a 1st sentence, and after that you go with your sentence but make it more open like you are unsure at this point and speculate a little]

For the realization of this definition, the control architecture of an agent is of main concern. The underlying classification consists of deliberative and reactive architectures \cite{behaviour-based-methods}. The first one involves extensive planning relying on a detailed world model and while performing well in highly predictable and describable environments fails to act efficiently in real-time as well as to react to unforeseen occurrences. \par 
The opposite direction is taken by reactive architecture, which aims to tightly couple sensory input with resulting actions by minimizing the internal world representation and making decisions basing solely on the current environment state \cite{arkin-1998}. While the reactive control architecture successfully solves the problem of robust real-time action selection in a changing environment, it does not guarantee proper goal pursuance and even approximated optimality of a solution.\par
A particular, behaviour-based kind of reactive approach aims to enhance goal-orientedness while still remaining fast and highly adaptive. First introduced in \cite{maes}, it utilizes the concept of a network of selectable behaviours  and connections between them that allow for activation spreading, which in turn is responsible for action selection. While this intuitive approach is very promising in the aspects of situation relevance and balance between adaptivity and goal pursuance, questions of optimal activation computation and its evaluation remain unanswered.\par
Also as an attempt to address the problems of both former architectures a third, hybrid one, emerged, combining goal pursuance of deliberative, plan based architecture with adaptivity and robustness of the reactive one, arranged in three layers: planning, control execution and reactive \cite{arkin-1998}. While the deliberative layer has the properties of completeness and correctness, it is  not the case for the reactive architecture. Improvements are still needed in the areas of activation spreading predictability and parameters tuning in order to adjust the algorithm for a particular use case. \par
This thesis focuses on the analysis and improvement of the current activation spreading process being followed by the evaluation of the revised process. As first the current state of scientific research in the field is examined together with existing communities and platforms for the development of robotic systems and alternative solutions in the area of decision making are discussed. Hereafter in chapter 3 the main objectives for the thesis are derived from the research. In chapter 4 the approach to reach the defined goals is described. Consequently, task packages for the thesis are defined in chapter 5. Finally, chapter 6 presents the estimated time frames for the defined tasks. 
\end{comment}
\begin{comment}
Länge: ca. 1 - 5. Seiten\\\\

\noindent Aufbau:
\begin{itemize}
	\item Motivation der Arbeit /Problem
	\item Ansatz der Lösung / Ziele
	\item Struktur der Arbeit / Vorgehen
\end{itemize}
Die Einleitung dient dazu, beim Leser Interesse für das Thema der Arbeit zu wecken, das behandelte Problem aufzuzeigen und den zu seiner Lösung eingeschlagenen Weg zu beschreiben. In diesem Kapitel wird die mit dem Betreuer/Professor besprochene Aufgabenstellung herausgearbeitet und für einen potentiellen Leser "spannend" dargestellt.\\\\

\noindent Motivation:\\
In der Motivation wird dargestellt, wieso es notwendig ist, sich mit dem in der Arbeit identifizierten und behandelten Problem zu beschäftigen. Zur Entwicklung der Motivation kann eine dem Leser bekannte Problematik aufgegriffen und dann die Problemstellung hieraus abgeleitet werden. Der Betreuer unterstützt die Entwicklung der Motivation, indem er bei der Einordnung der Arbeit in ein größeres Problemgebiet hilft.\\\\

\noindent Häufige Fehler:
\begin{itemize}
	\item Zu allgemeine Motivation, Problemstellung und -abgrenzung. Die Problemstellung beginnt mit der Einordnung in ein thematisches Umfeld und enthält sowohl die in der Arbeit angegangenen Problempunkte, als auch weitere, nicht behandelte Problempunkte. Eine Negativabgrenzung verhindert, dass beim Leser später nicht erfüllte Erwartungen geweckt werden.
Der Betreuer unterstützt die Eingrenzung der Problemstellung, indem er Hinweise auf abzugrenzende Punkte bzw. auszuschließende Punkte im Rahmen der Negativabgrenzung gibt.
\noindent Häufige Fehler:
	\item Keine klare Problemstellung und -abgrenzung
	\item Fehlen der Negativabgrenzung
\end{itemize}

\noindent Ziel der Arbeit:\\
Mit dem Ziel der Arbeit wird der angestrebte Lösungsumfang festgelegt. An diesem Ziel wird die Arbeit gemessen.
Der Betreuer sorgt dafür, dass das Ziel der Arbeit realisierbar und im Rahmen einer Diplomarbeit lösbar ist.\\\\
\noindent Häufige Fehler:
\begin{itemize}
	\item Kein klares Ziel
	\item Zu viele Ziele
\end{itemize}


\noindent Vorgehen:\\
Nachdem mit Problemstellung und Ziel gewissermaßen Anfangs- und Endpunkt der Arbeit beschrieben sind, wird hier der zur Erreichung des Ziels eingeschlagene Weg vorgestellt. Dazu werden typischerweise die folgenden Kapitel und ihr Beitrag zur Erreichung des Ziels der Arbeit kurz beschrieben. Die folgenden Kapitel sind ein {\em möglicher} Aufbau, Abweichungen können durchaus notwendig sein. Zur Darstellung des Vorgehens kann eine grafische Darstellung sinnvoll sein, bei der die einzelnen Lösungsschritte und ihr Zusammenhang dargestellt werden.
\end{comment}
