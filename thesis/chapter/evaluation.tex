\chapter{Evaluation}

\begin{comment}
Länge: ca 1-5 Seiten\\\\
Sind die gesteckten Ziele zur Problemlösung durch die Implementierung erreicht worden? Was kann die vorgestellte Lösung, was kann sie nicht. Des Weiteren gehören zu einer Implementierung auch immer Tests, ob die Implementierung erfolgreich war! Diese Tests/Versuche müssen auch dokumentiert werden. In diesem Kapitel sollte daher eine Beschreibung des Aufbaus und der Ergebnisse von Testläufen/Simulationen vorhanden sein. Ebenso sollte eine Interpretation der Ergebnisse die Tests abschließen. Es ist auch wichtig, nicht nur positive, sondern auch negative Ergebnisse zu dokumentieren und zu interpretieren.

Hier sollte auch ein Bezug zu ggf. aufgestellten Forschungsfragen gefunden werden.
\end{comment}

In this section, the process of evaluation of the implemented algorithm is presented. To achieve this, firstly, the simulation environment is defined including the \textit{USAR} domain, the robot and the objectives. Following, the process of simulation itself is described and the resulting scores, metrics and remarks presented. Finally the results are thoroughly discussed.

\begin{itemize}
    \item Construction of test scenarios for comparative evaluation with particular focus on exposure of the discrimination of behaviours under conditions discussed
    \item Execution of implemented activation spreading algorithms for the constructed test scenario.
    \item The evaluation consists of two main parts. As first constructed test scenarios are executed multiple times with each of the implemented activation algorithm alterations for more robust activation distribution results. During the execution the state of the behaviour network and the environment in each step is recorded. Following the execution, results are analyzed and discussed in terms of right goal pursuance, quality of the solution (i.e. is the resulting behaviour sequence the optimal one in given environment state) and adaptiveness (i.e. how well does the algorithm cope with finding alternative behaviours sequences in changing conditions). Relevant metrics such as number of steps or behaviour switches and domain specific metrics are also considered. Subsequently, the distribution of the activation values in the network throughout the execution of test scenario is visualized and examined with exploration of potential undesirable outliers as behaviours reaching disproportional activation values to their situation dependent importance for the goal pursuance.
\end{itemize}
\section{Simulation environment}
Implementation is followed by the construction of a realistic 3D simulated environment in which alternative versions of decision-making algorithm are tested, with Morse simulator \cite{morse} as one of possible tools to realize it. To fully explore the benefits of behaviour based approach as well as its potential flaws, such an environment has to guarantee multiple qualitatively different ways of achieving same goals, behaviours varying in the number of incoming and outgoing activation links in conjunction with occurrence of both achievement and maintenance goals. Such a variety of possible actions together with more possible ways to achieve or maintain goals is offered by the area of \textit{Urban Search And Rescue robotics} (USAR) \cite{usar} where reactivity and goal pursuance are equally crucial.
In particular the simulation used for evaluation in this thesis focuses on modelling the action selection problem faced by search and rescue robots, that are applicable among others in road incidents, natural catastrophes and bomb explosions. Critical abilities of such robots include movement, obstacle shifting and avoidance, grasping, recognition and examination of living creatures as well as first aid. Their main tasks are exploring particular areas, rescuing people and managing their health condition, securing the infrastructure and managing the energy supply. In order to correctly and efficiently perform these tasks robots have a variety of sensors including audio, video, vicinity, temperature and chemical among others at their disposal allowing them for detailed perception of environment. 
\par The described use case serves as a basis for construction of the simulation. Its environment consists of a finite operational area, an abstraction of an indoor space of a collapsed building, on which a mobile wheeled robot with grasping and healing capability navigates between obstacles or moves them and searches for objectives. There are four main achievement goals: exploration of checkpoints given a priori, localization of a trapped, endangered victim, getting to him or her in order to analyze the person's health condition and heal him or her if necessary as well as repairing a damaged part of infrastructure. Since all actions needed to undertake in order to achieve these goals cause energy consumption and the robot has a battery with limited capacity, the fifth, maintenance goal emerges in form of keeping the energy reserve above a fixed threshold allowing the robot for safe return from the danger zone. The defined goals set the granularity level for behaviours which the robot has at its disposal. The set of actions possible to undertake includes moving to a particular checkpoint/location, grasping and moving obstacles, scanning the vicinity in search for victims, healing endangered victims as well as repairing a damaged infrastructure part. To enable adequate perception of the environment the robot has position, proximity and battery sensors together with an abstract complex sensor which enables detecting victims and providing first aid. Encoding the dependencies between the sensors, behaviours and goals results in a behaviour network that serves as the basis for evaluation.
\begin{itemize}
    \item MORSE
    \item USAR domain
\end{itemize}
\section{Execution}

\section{Results}