\chapter{Problem definition}
\begin{comment}
The resulting challenges and issues extracted in chapter 2 clearly show that there is still room for improvement in the reactive layer of RHBP. In-depth analysis and evaluation of the decision-making algorithm currently used in RHBP as well as exploration of possibilities for improvement of its adaptivity and robustness form the main objectives of this thesis. \par 
In particular, the alternatives proposed by \cite{tyrrell} and \cite{dorer} are examined. Both aim to eliminate suboptimal spreading of activation in networks in particular edge cases, that however cannot be excluded from occurring more often  than just occasionally in real life scenarios. Another objective of the REASM proposed by \cite{dorer} is exploiting information gained from continuous states as well as the ability to express situation dependent relevance of goals. These alternatives will be discussed in detail, implemented and evaluated basing on test scenarios constructed in such a way that aforementioned edge cases occur. \par 
Furthermore, the activation distribution among behaviours throughout the repeated test scenario execution will be examined and the influence of the distribution uniformity on the properties of the decision-making process will be discussed as well as the causes of potential anomalies during the  process will be detected and explored.
\end{comment}
\begin{comment}
Länge: ca. 5 - 15 Seiten\\\\
Das Kapitel Problemanalyse dient dazu, das in der Einleitung identifizierte und eingegrenzte Problem auf seine Ursachen zurückzuführen und so Lösungsmöglichkeiten zu entwickeln. Hierdurch wird die Problembezogenheit der entwickelten Lösung sichergestellt. Wenn möglich, ist durch eine Literaturrecherche nachzuweisen, dass bisher keine geeigneten Lösungen existieren. Der Betreuer hilft bei der Entscheidung, ob die Problemanalyse ausreichend tief erfolgt ist. Häufig führt eine hinreichend genaue Problemanalyse zu präziseren und damit kürzeren Lösungskonzepten.\\\\

\noindent Kapitelzusammenfassung am Ende:\\
Der Übergang von der Problemanalyse zur Konzeptentwicklung stellt eine wichtige Nahtstelle innerhalb der Arbeit dar, da von einer betrachtend-analysierenden Perspektive auf eine konstruktiv-kreative Perspektive gewechselt wird. Daher empfiehlt es sich, an dieser Stelle die Ergebnisse der Problemanalyse zusammenzufassen.
\end{comment}
