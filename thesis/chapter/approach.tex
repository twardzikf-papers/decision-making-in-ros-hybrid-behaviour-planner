\chapter{Approach}
\begin{comment}
To reach the defined goals, the implementation of alternative activation algorithms in the RHBP is of main concern as the first step. It is followed by the construction of suitable simulated environment  in order to test the implementations in scenarios which expose mentioned problems of discrimination of behaviours by the number of inputs and outputs as well as of activation distribution throughout the network. As next the constructed test scenarios are executed and the state of the behaviour network is recorded with particular focus on changing activation values. Consequently the evaluation of the results follows together with discussion and outlook.\par 
The implementation encompasses three particular modifications of activation formula, proposed by \cite{tyrrell} that focus on the division by the number of inputs and outputs and its impact on the activation calculation throughout the network. Since the omission of the division by the number of outputs, which causes discrimination of behaviours having many outgoing activation links, is proven to not have any negative effects, it is included in each following alteration of activation algorithm. The omission of the division by the number of inputs also eliminates the problem of discrimination of the behaviours with multiple incoming activation links but causes the opposite problem of the preference of these nodes. To fully explore the impact of this anomaly three proposed versions are included: with and without the division by the number of inputs as well as taking the average value of both first variants.  
\par Moreover the REASM proposed in \cite{dorer} is analyzed and implemented. As an extension that does not modify the architecture of the Behaviour Network in RHBP, two particular changes from REASM are considered for implementation. Firstly, the introduction or relevance condition for goals potentially allows for better control of activation and inhibition from goal basing on current environment state. Secondly, the aggregation function of all sources (excluding activation from plan) of activation in the network is changed from sum to maximum. As the result the final activation of each behaviour for each goal in each step equals just the highest value of all activation sources and thus prevents any confluence of activation within a behaviour from the same goal. 
\par Implementation is followed by the construction of a realistic 3D simulated environment in which alternative versions of decision-making algorithm are tested, with Morse simulator \cite{morse} as one of possible tools to realize it. To fully explore the benefits of behaviour based approach as well as its potential flaws, such an environment has to guarantee multiple qualitatively different ways of achieving same goals, behaviours varying in the number of incoming and outgoing activation links in conjunction with occurrence of both achievement and maintenance goals. Such a variety of possible actions together with more possible ways to achieve or maintain goals is offered by the area of \textit{Urban Search And Rescue robotics} (USAR) \cite{usar} where reactivity and goal pursuance are equally crucial.
\par In particular the simulation used for evaluation in this thesis focuses on modelling the action selection problem faced by search and rescue robots, that are applicable among others in road incidents, natural catastrophes and bomb explosions. Critical abilities of such robots include movement, obstacle shifting and avoidance, grasping, recognition and examination of living creatures as well as first aid. Their main tasks are exploring particular areas, rescuing people and managing their health condition, securing the infrastructure and managing the energy supply. In order to correctly and efficiently perform these tasks robots have a variety of sensors including audio, video, vicinity, temperature and chemical among others at their disposal allowing them for detailed perception of environment. 
\par The described use case serves as a basis for construction of the simulation. Its environment consists of a finite operational area, an abstraction of an indoor space of a collapsed building, on which a mobile wheeled robot with grasping and healing capability navigates between obstacles or moves them and searches for objectives. There are four main achievement goals: exploration of checkpoints given a priori, localization of a trapped, endangered victim, getting to him or her in order to analyze the person's health condition and heal him or her if necessary as well as repairing a damaged part of infrastructure. Since all actions needed to undertake in order to achieve these goals cause energy consumption and the robot has a battery with limited capacity, the fifth, maintenance goal emerges in form of keeping the energy reserve above a fixed threshold allowing the robot for safe return from the danger zone. The defined goals set the granularity level for behaviours which the robot has at its disposal. The set of actions possible to undertake includes moving to a particular checkpoint/location, grasping and moving obstacles, scanning the vicinity in search for victims, healing endangered victims as well as repairing a damaged infrastructure part. To enable adequate perception of the environment the robot has position, proximity and battery sensors together with an abstract complex sensor which enables detecting victims and providing first aid. Encoding the dependencies between the sensors, behaviours and goals results in a behaviour network that serves as the basis for evaluation.
\par The evaluation consists of two main parts. As first constructed test scenarios are executed multiple times with each of the implemented activation algorithm alterations for more robust activation distribution results. During the execution the state of the behaviour network and the environment in each step is recorded. Following the execution, results are analyzed and discussed in terms of right goal pursuance, quality of the solution (i.e. is the resulting behaviour sequence the optimal one in given environment state) and adaptiveness (i.e. how well does the algorithm cope with finding alternative behaviours sequences in changing conditions). Relevant metrics such as number of steps or behaviour switches and domain specific metrics are also considered. Subsequently, the distribution of the activation values in the network throughout the execution of test scenario is visualized and examined with exploration of potential undesirable outliers as behaviours reaching disproportional activation values to their situation dependent importance for the goal pursuance.\par
\end{comment}
\begin{comment}
Länge: ca. 5 - 15 Seiten\\\\
Im Lösungskonzept wird auf konzeptueller Ebene der Weg zur Lösung der identifizierten Probleme beschrieben. Ausgangspunkt sind die Erkenntnisse der vorangegangenen Problemanalyse. Wichtig ist hierbei die Herausstellung des erzielten Neuigkeits- und Innovationswertes im Bezug auf den bisherigen Stand der Technik/Wissenschaft. Grundlage hierfür ist ebenfalls die im vorangegangenen Kapitel durchgeführte Problemanalyse. Im Lösungskapitel werden noch keine umsetzungsspezifischen Details angeführt, dies ist Aufgabe des folgenden Kapitels. Eine typische Gliederung für die Darstellung des Lösungskonzepts ist das Aufgreifen der im vorangegangenen Kapitel identifizierten Problembereiche. Der Betreuer berät bei der Darstellung des Lösungskonzepts.\\\\

\noindent Häufige Fehler:
\begin{itemize}
	\item Lösungskonzept passt nicht zum Ziel
	\item Lösungskonzept enthält Bestandteile der Umsetzung
\end{itemize}

\noindent Kapitelzusammenfassung am Ende:\\
Eine Zusammenfassung erleichtert es dem Leser, die erarbeitete Lösung zu erfassen.
\end{comment}